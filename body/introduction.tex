\section{Introduction}
With the development of cloud computing and big data technologies,
more and more applications are being developed for cloud platforms
and third-party data centers. However, cloud platform applications
often receive various threats, especially the leakage of critical data,
which makes many companies reluctant to use the cloud platform.
Fortunately, trusted computing prevents confidential violation and
protects their applications running on shared servers.
In recent years, cloud computing service companies supported confidential
computing and provided corresponding Trusted Execution Environment (TEE).
For example, Amazon Nitro system \cite{AmazonNitro} uses hardware-based memory
isolation to protect data, and Azure \cite{AzureDocs} provides computing environments
for Intel SGX and AMD SEV-SNP and confidential container computing.

How to migrate original applications to a trusted environment becomes another
vital issue. Although both SGX and TrustZone provide their SDKs,
it requires programmers to divide the trusted and untrusted parts.
Glamdring \cite{Lind2017GlamdringAA} can split the code into untrusted and
trusted parts based on tagged data. Occlum \cite{Shen2020OcclumSA} brings
the LibOSes into SGX to support legacy programs. SCONE \cite{Arnautov2016SCONESL}
provides a secure C standard library interface that allows applications to
run in secure containers.
These solutions usually require source code and recompile (Glamdring) or
the entire program to be run in an enclave (Occlum and SCONE).
For those programs where the source code is not available,
using Glamdring is not feasible.
In addition, putting the entire program into the enclave increases
the size of the TCB and expands the attack surface,
which is also an unacceptable solution.

Another obstacle is the performance of trusted computing.
Switchless Calls \cite{Tian2018SwitchlessCM} changes synchronous execution
to asynchronous execution, which can reduce enclave switching.
VAULT \cite{Taassori2018VAULTRP} introduces a variable arity unified tree (VAULT),
which compresses the Enclave Page Cache (EPC) and saves overhead.
Other studies have focused more on reconfigurable trusted hardware,
such as TEEOD \cite{Pereira2021TowardsAT} and BYOTee \cite{Armanuzzaman2022BYOTeeTB},
which use heterogeneous SoC or FPGA to implement some new features.
These hardware-related studies are more concerned with new features than
performance optimization. CRONUS \cite{Jiang2022CRONUSFS} gives some suggestions
on how to modify the hardware to speed up trusted computing but does not
implement them.

Therefore, it is a tricky problem to efficiently migrate legacy programs to TEEs,
especially those for which source code is unavailable. This research
will first explore the algorithms for finding confidential code and data
for legacy programs (Section \ref{sec:ToolToAnalyze}). Then we will use these results
to guide the rewriting tool to modify legacy programs into software that
supports trusted computing (Section \ref{sec:ToolToRewrite}). We intend to have
our tools support a variety of legacy software, including distributed programs
(Section \ref{sec:ToolToDistributedSystem}), and still maintain high execution efficiency
with the hardware-assisted (Section \ref{sec:ToolWithOptimization}).

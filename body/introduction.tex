\section{Introduction}
With the development of cloud computing and big data technologies,
more and more applications are being developed for cloud platforms
and third-party data centers. However, cloud platform applications
often receive various external and insider attacks, which could
incur immense leakage of sensitive user data.
For example, hackers stole 1 billion records of Chinese citizens from police \cite{ChinaHackPolice}.
Also, Facebook grossly misuses user data for U.S. election \cite{Facebook}.
Fortunately, trusted computing prevents confidential violations and
protects their applications running on shared servers.
In recent years, cloud computing service companies supported confidential
computing and provided corresponding Trusted Execution Environment (TEE).
For example, Amazon Nitro system \cite{AmazonNitro} uses hardware-based memory
isolation to protect data, and Azure \cite{AzureDocs} provides computing environments
for Intel SGX and AMD SEV-SNP and confidential container computing.

However, it is non-trivial to execute general programs, typically legacy (off-the-shelf)
applications in a trusted execution environment with security guarantee.
The first solution is to \textbf{rewrite the source code} and recompile it.
Although both SGX and TrustZone provide SDKs to assist programming, it requires
programmers with substantial expertise in TEE to divide the trusted and untrusted parts,
which is a tedious and error-prone task \cite{dong2021research}. Several automated tools
are used to assist in rewriting the source code to make this process easier.
For example, Glamdring \cite{Lind2017GlamdringAA} can automatically
split the code into untrusted and trusted parts based on tagged data.
Unfortunately, we often need to address software for which source code is not available,
such as commercial databases and accounting modules. For those programs where the
source code is not available, using Glamdring is not feasible.
Another solution is to use \textbf{shielding} to support general binary (X86) programs.
For example, Occlum \cite{Shen2020OcclumSA} brings the LibOSes
into SGX to support general programs without modification. SCONE \cite{Arnautov2016SCONESL}
provides a secure C standard library interface that allows applications to
run in secure containers. They both require an entire program to be run in an enclave.
Although shielding does not require source code,
putting the entire program into the enclave can increase the size of the
Trusted Computing Base (TCB), such as Graphene \cite{Tsai2017GrapheneSGXAP}
adding 10 kLoC to TCB.
The growth of the TCB size leads to the expansion of the attack surface,
which is also an unacceptable solution.

Worse, even when applications can be executed transparently and safely within a TEE,
these applications often experience severe performance issues.
These problems are usually caused by frequent context switches and memory limitations.
Switchless Calls \cite{Tian2018SwitchlessCM} changes synchronous execution
to asynchronous execution, which can reduce enclaves switching.
VAULT \cite{Taassori2018VAULTRP} introduces a variable arity unified tree (VAULT),
which compresses the Enclave Page Cache (EPC) and saves overhead.
Other studies have focused more on reconfigurable trusted hardware,
such as TEEOD \cite{Pereira2021TowardsAT} and BYOTee \cite{Armanuzzaman2022BYOTeeTB},
which use heterogeneous SoC or FPGA to implement some new features.
These hardware-related studies are more concerned with new features than
performance optimization. CRONUS \cite{Jiang2022CRONUSFS} gives some suggestions
on modifying the hardware to speed up trusted computing but does not
implement them.
The vast overhead is due to the complex hardware design, such as enclave switching
and encryption overhead, which can introduce thousands of clock cycles of performance
loss \cite{Weichbrodt2018sgxperfAP}. 
Despite tremendous work in optimizing performance, most of their work is based on the
software level, which only mitigates this overhead.

To tackle the aforementioned ease-of-use, security, and performance issues, this proposal explores
algorithms and system designs for executing general programs within TEE securely and efficiently.
We will use Programming Language (PL) and software co-design as well as hardware and software co-design
to tackle the problem.
First, we will explore the algorithms for finding confidential code and sensitive data of
general programs (Section \ref{sec:ToolToAnalyze}). Then we will use these results
to guide our new compiler to make general programs support trusted computing through
code rewriting, binary translation or emulation (Section \ref{sec:ToolToRewrite}).
By doing so, this approach is general enough to support both general programs running within
one process and using a distributed manner (Section \ref{sec:ToolToDistributedSystem}).
Finally, we will create performance models for existing TEE hardware (e.g., TEE context switches)
to capture performance bottlenecks by inserting statistical code at compile-time,
which help the profiler to detect performance bottlenecks at runtime to generate
performance-optimized TEE code. By doing so, we will propose new hardware primitives
to accommodate the performance model and profiler.
Moreover, for the performance bottlenecks identified by the profiler, we are ready to
perform hardware and software co-optimization to achieve better performance
(Section \ref{sec:ToolWithOptimization}).

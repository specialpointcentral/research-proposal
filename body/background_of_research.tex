\section{Background of Research}
\subsection{Trusted Computing and TEE}
Various encryption and authentication methods (e.g., TLS and file disk encryption)
are often used to prevent confidential data loss, theft or corruption.
However, relying on software alone to protect confidential data has many problems,
such as software vulnerabilities and reverse engineering cracking \cite{Zimba2021ARC}.
So, it is helpful to use Trusted Execution Environment (TEE) to protect this encryption
software and data, which provides an environment shielded from outside interference and
the necessary mechanisms to build secure and sensitive applications.

Isolation and attestation are two critical components of trusted computing.
The code, data and runtime state in the enclave are secured by \textbf{isolating} memory,
thus preventing access or corruption from outside the enclave.
The legitimacy of the remote user is guaranteed through \textbf{attestation},
including the software state measurement and hardware key signing.

Intel Software Guard Extensions (SGX) is a set of security architecture extensions
\cite{McKeen2013InnovativeIA}.
It provides the enclave environment that prevents other software from accessing
the code and data inside an enclave. Also, when data leaves the enclave and is written
into the memory, the data will be automatically encrypted.

ARM TrustZone uses a different approach to TEE by introducing a secure world,
which is a new execution environment besides the normal world in the processor
\cite{Mukhtar2019ArchitecturesFS}. The secure world has multiple privilege levels,
just like a virtual machine (VM), which allows an entire trusted software stack to
be implemented.

% new Arch
% Due to overly complex operations and unacceptable hardware overhead, Intel started to move
% towards Trust Domain Extensions (TDX) \cite{Sahita2021SecurityAO, Sardar2021DemystifyingAI},
% a new trusted computing architecture that introduces a separate trusted hypervisor. The interaction
% between trusted virtual machines and external intrusted environments should be checked by the
% security check module Shim.

% Since TrustZone lacks confidentiality support, ARM v9 proposes Confidential Compute Architecture (CCA).
% CCA \cite{CCA} differs from TrustZone, directly supports in-memory confidentiality capabilities in hardware,
% and protects users' confidential data.

\subsection{Binary rewriting and binary translation}
Binary rewriting is a technique for modifying or translating the original binary code
without having the source code. According to the characteristics, they can be divided
into four categories: static, dynamic, minimal-invasive and full-translation.

Static binary rewriting can use the existing information, such as static data flow analysis and
symbol table information, to optimize or enhance existing programs \cite{10.1145/2629335, Schwarz2007PLTOAL}.
Dynamic binary rewriting performs alterations during execution, which can be used for
performance analysis \cite{Luk2005PinBC} and hot code patching \cite{Bruening2003AnIF}.
Minimal-invasive rewriting is based on branch granularity. It will perform additional instruction
at the original location by rewriting into branch instructions. This is often used to add a new
function to the original program \cite{Federico2017revngAU}.
Full-translation rewriting can convert binaries at any instruction and usually lift the original
binary code into intermediate translation representations. Some open-source tools,
like QEMU \cite{Bellard2005QEMUAF} and Valgrind \cite{Nethercote2007ValgrindAF},
use full-translation for binary rewriting.

% difference between rewriting and translation?
% same target/cross-instruction?
% summary

\subsection{Distributed system in Trusted computing}
% spark/Hadoop/Ceph: threat
With the rise of cloud computing and the increase in data sets in recent years,
more and more occasions require the use of distributed systems.
While distributed systems, such as Hadoop and Spark, are facing an increasing number
of threats.

However, the existing TEEs, including SGX and TrustZone, only protect integrity
and confidentiality within a single instance, which cannot protect multiple
instances programs, such as distributed systems.
Without any protection, communications between multiple instances can be easily
tampered with, leading to integrity and rollback attacks.
Although there is substantial research on distributed systems, they do not propose a
distributed integrity protocol that is both general and efficient.

Distributed MapReduce system VC3 \cite{Schuster2015VC3TD} was proposed in 2015,
which keeps the code and data confidential, and ensures the correctness and completeness of
the results. SGX-PySpark \cite{Quoc2019SGXPySparkSD} was implemented to protect confidential data
with the help of TEE in 2019.

For other systems, such as database, EnclaveDB \cite{Priebe2018EnclaveDBAS} uses SGX to
protect the database engine and ensure high performance.
EncDBDB \cite{Fuhry2021EncDBDBSE} is optimized for column-oriented in-memory databases,
also uses SGX for data security.

In recent years, heterogeneous computing systems, such as Computation Storage Architectures (CSA),
have also faced data security issues.
IronSafe \cite{Unnibhavi2022SecureAP} provides a secure processing system for heterogeneous
computing storage architectures using a hardware-assisted trusted execution environment.

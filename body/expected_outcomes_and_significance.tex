\section{Expected Outcomes and Significance}
In order to evaluate the effectiveness of our work, as we mentioned in section
\ref{sec:ToolToAnalyze}, we can use a series of open-source test cases and benchmarks,
such as SPEC, UnixBench. 
We will also consider using SGXGauge \cite{Kumar2022ACB}, a test suite specifically
designed to test the performance of trusted systems, to measure the performance
of our tool.
Besides that, we are able to compare our tools with other cross-platform TEEs.
For example, we can use HyperEnclave \cite{Jia2022HyperEnclaveAO}, TrustVisor
\cite{McCune2010TrustVisorET}, Occlum \cite{Shen2020OcclumSA} and Graphene
\cite{Tsai2017GrapheneSGXAP} as baseline systems for performance analysis.

Although these cross-platform TEEs propose various solutions to address the
legacy programs, most are modifying libraries and putting them into the
enclave as a whole. This could lead to the expansion of the TCB and lead
to an expansion of the attack surface.

Our solution uses binary rewriting to place the analyzed confidential code and data
into enclaves, which both hardening the legacy problems and reducing the size of TCB.
We anticipate our tools for legacy programs can reach the following goals:
\begin{enumerate}
    \item Analysis tool can efficiently and accurately identify vulnerable code and data.
    Using analysis tool alone can help developers find software weaknesses and flaws.
    \item Binary rewriting tool can rewrite legacy programs into enclave-protected programs
    with the help of analysis results.
    \item The extension of tools support distributed legacy programs.
    \item Based on the performance data, we design or modify the hardware.
    With hardware assisted, we expect the performance can exceed other cross-platform TEEs,
    or even approach the performance of native programs.
\end{enumerate}
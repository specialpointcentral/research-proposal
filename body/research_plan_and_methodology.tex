\section{Research Plan and Methodology}
\subsection{Design a tool to analyze legacy programs}
In order to complete the migration of a legacy system, the first step is
to analyze the program and identify the code and data that needs to be migrated.
There are three parts we need to consider: data analysis, recognition algorithm
and recognition accuracy.

The first is \textbf{data analysis}. The purpose of the analysis is to provide
information about binaries so that the subsequent identification process can
proceed smoothly.
There are many similar works in the field of binary rewriting to explore
the accuracy of binary analysis.
BIRD \cite{Nanda2006BIRDBI} uses a combination of static and dynamic identification
methods to improve the accuracy of the analysis. However, for dynamic disassembly,
some trapped instruction need to be inserted which is not a good choice for us.
Other papers point out that static disassembly can also achieve good results
\cite{Andriesse2016AnIA}. So we prepare to implement the static analysis tool first
and then decide whether to add dynamic analysis based on the accuracy of the analysis.

The next is the \textbf{recognition algorithm}.
Moat \cite{Sinha2015MoatVC} is a detection tool designed by Berkeley that uses
automatic theorem proving and information flow analysis methods to discover the
possibility of applications leakage of secret information in the SGX region by
analyzing the assembly language level of the program.
Our work can be based on Moat, from which we can extract effective identification
and verification algorithms and use them in our analysis tools. 

In order to evaluate \textbf{recognition accuracy}, we will consider it in two parts.
\textit{Coverage of analysis}:
We will use different test cases to see how well the overall analysis is covered.
Since we have access to the source code of these test programs,
the coverage of the analysis can be measured by some tools such as gcov \cite{GCOV}
and QEMU \cite{Bellard2005QEMUAF}.
\textit{Correctness of the analysis}:
We also use the test cases available in the source code to verify correctness.
We would like to compare the results of our analysis tool with the results of the source code
after automatic analysis by Glamdring \cite{Lind2017GlamdringAA} in order to obtain
an accurate analysis.

\subsection{Design a binary rewriting tool to protect confidential code and data}

\subsection{Extend the binary rewriting tool into distribute system}

\subsection{Optimize the above system by software-hardware co-optimization}


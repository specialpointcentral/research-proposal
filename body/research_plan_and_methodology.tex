\section{Research Plan and Methodology}
\subsection{Design a module to analyze general programs}
\label{sec:ToolToAnalyze}
In order to complete the migration of a general programs, particularly legacy system,
the first step is to analyze the program and identify the code and data that needs to
be migrated. We need to consider three parts: dataflow analysis, recognition algorithm
and recognition accuracy.

The first is \textbf{dataflow analysis}. The purpose of the analysis is to provide
information about the binary program so that the subsequent identification process
can proceed smoothly.
There are many similar works in the field of binary rewriting to explore
the accuracy of binary analysis.
BIRD \cite{Nanda2006BIRDBI} uses a combination of static and dynamic identification
methods to improve the accuracy of the analysis. However, some trapped instructions
need to be inserted during dynamic disassembly, which is not a good choice for us.
Other papers indicate that static disassembly can also achieve good results
\cite{Andriesse2016AnIA}. So we prepare to implement the static analysis module first
and then decide whether to add dynamic analysis based on the accuracy of the analysis.

The next is the \textbf{recognition algorithm}.
Moat \cite{Sinha2015MoatVC} is a detection tool designed by Berkeley that uses
automatic theorem proving and information flow analysis methods to discover the
possibility of application leakage of secret information in the SGX region by
analyzing the assembly language level of the program.
Our work can base on Moat, and extract effective identification and verification
algorithms from Moat, which can be used in our analysis module.

To evaluate \textbf{recognition accuracy}, we will consider it in two parts.
\textit{Coverage of analysis}:
We will use different test cases to see how well the overall analysis is covered.
Since we have access to the source code of these test programs,
the coverage of the analysis can be measured by some tools such as Gcov \cite{GCOV}
and QEMU \cite{Bellard2005QEMUAF}.
\textit{Correctness of the analysis}:
We also use open-source test cases to verify correctness.
We would like to compare the results of our analysis tool with the results of the
source code after automatic analysis by Glamdring \cite{Lind2017GlamdringAA} to obtain
an accurate analysis.

\subsection{Design a new compiler to protect confidential code and data}
\label{sec:ToolToRewrite}
With the help of an analysis module, it is easy to obtain the code segments that need
to be protected and the memory areas where vital data is located.

\textbf{Code protection}. We can use the minimal-invasive translation method for
code segments that need to be protected. Similar to the rev.ng \cite{Federico2017revngAU}
and pin \cite{Luk2005PinBC}, we can insert required functions before and
after the code segments. We insert the enclave's entry code and enclave's parameters passing code
before the segments. Also, enclave's return parameters can be built at the end of the segments.
Many more details should be noticed and considered, which need to be
discovered and resolved during research.

\textbf{Data protection}. Data protection is more complicated than code protection,
especially for global variables.
For local variables, we can analyze them, get the program boundary,
and put the variables as well as code into enclaves for protection.
PtrSplit \cite{Liu2017PtrSplitSG} focuses on identifying C/C++ pointers
that prevent the generation of partition boundaries.
But for global variables, there is no good solution for now. However,
global variables are often not recommended for a highly cohesive and low-coupling system
\cite{GlobalVariablesAreBad, GlobalVariablesAreEvil},
so dropping protection for this part when it cannot be solved is generally not a big deal.

\subsection{Extend the binary rewriting compiler into distributed system}
\label{sec:ToolToDistributedSystem}
Our work will explore two areas related to distributed systems, how to support general distributed
programs and enable our system to run on distributed platforms.

\textbf{Support distributed programs}.
These distributed programs tend to have more complex features than ordinary programs.
For example, OpenMP \cite{Dagum1998OpenMPAI} and MPI+OpenMP \cite{Klinkenberg2020CHAMELEONRL}
will use mechanisms such as semaphores, message communication, etc., which cause problems
for both the identification and transformation of our system.
We will explore and tackle these challenges during our research process.

\textbf{Compiler run on distributed systems}.
How to run our system on a distributed platform is another complex work.
DQEMU \cite{Zhao2020DQEMUAS} achieves a distributed dynamic binary translation system.
It discusses the implementation issues and performance optimization, including
data coherence protocol, locking mechanism, system calls, and remote thread migration.
We can take the idea of DQEMU and modify our system to run on a distributed platform.

\subsection{Optimize the system by software-hardware co-optimization}
\label{sec:ToolWithOptimization}
Whether running secret code in enclaves or using rewriting compiler for transformation,
they both introduce a significant performance overhead.
Many studies investigated the overhead of trusted computing and
the corresponding optimization method, including avoiding enclave switches
\cite{Tian2018SwitchlessCM} and reducing page swaps \cite{Orenbach2017EleosEO, Taassori2018VAULTRP}.
But these optimizations may be challenging to implement in our system because it is hard
to change the program's original behavior.
In addition, binary rewriting also faces significant performance overhead, and existing
software optimizations are limited in dealing with these issues \cite{Kim2003HardwareSF}.

Some studies have proposed several ideas for hardware-assisted acceleration, such as
shared memory \cite{Jiang2022CRONUSFS}.
So our future work will summarize the existing optimization and search the performance
bottlenecks through performance analysis tools, such as Perf, VTune and sgx-perf \cite{Weichbrodt2018sgxperfAP}.
After we obtain the bottlenecks of the performance, we can summarize the common characteristics
and design or modify some hardware modules to speed up our system and programs, similar to
the PIE \cite{Schneider2021PIEAP}.
\section{Research Plan and Methodology}
\subsection{Design a tool to analyze legacy programs} \label{sec:ToolToAnalyze}
In order to complete the migration of a legacy system, the first step is
to analyze the program and identify the code and data that needs to be migrated.
We need to consider three parts: data analysis, recognition algorithm
and recognition accuracy.

The first is \textbf{data analysis}. The analysis aims to provide
information about binaries so that the subsequent identification process can
proceed smoothly.
There are many similar works in the field of binary rewriting to explore
the accuracy of binary analysis.
BIRD \cite{Nanda2006BIRDBI} uses a combination of static and dynamic identification
methods to improve the accuracy of the analysis. However, for dynamic disassembly,
some trapped instructions need to be inserted which is not a good choice for us.
Other papers point out that static disassembly can also achieve good results
\cite{Andriesse2016AnIA}. So we prepare to implement the static analysis tool first
and then decide whether to add dynamic analysis based on the accuracy of the analysis.

The next is the \textbf{recognition algorithm}.
Moat \cite{Sinha2015MoatVC} is a detection tool designed by Berkeley that uses
automatic theorem proving and information flow analysis methods to discover the
possibility of application leakage of secret information in the SGX region by
analyzing the assembly language level of the program.
Our work can be based on Moat, from which we can extract effective identification
and verification algorithms and use them in our analysis tool. 

To evaluate \textbf{recognition accuracy}, we will consider it in two parts.
\textit{Coverage of analysis}:
We will use different test cases to see how well the overall analysis is covered.
Since we have access to the source code of these test programs,
the coverage of the analysis can be measured by some tools such as Gcov \cite{GCOV}
and QEMU \cite{Bellard2005QEMUAF}.
\textit{Correctness of the analysis}:
We also use open-source test cases to verify correctness.
We would like to compare the results of our analysis tool with the results of the
source code after automatic analysis by Glamdring \cite{Lind2017GlamdringAA} to obtain
an accurate analysis.

\subsection{Design a binary rewriting tool to protect confidential code and data}
With the help of an analysis tool, it is easy to obtain the code segments that need
to be protected and the memory areas where vital data is located.

\textbf{Code protection}. We can use the Minimal-invasive translation method for
code segments that need to be protected. Similar to the rev.ng \cite{Federico2017revngAU}
and pin \cite{Luk2005PinBC}, we can insert the required functions before and
after the code segments. We insert the enclave's entry code and enclave's parameters passing code
before the segments. Also, enclave's return parameters can be built at the end of the segments.
There should be many more details to note and consider here that need to be
discovered and resolved during research.

\textbf{Data protection}. Data protection is more complicated than code protection,
especially for global variables.
For local variables, we can analyze them, get the program boundary,
and put the variables as well as code into enclaves for protection.
PtrSplit focuses on C/C++ pointers and identifies pointers that block
the generation of partition boundaries \cite{Liu2017PtrSplitSG}.
But for global variables, there is no good solution for now. However,
global variables are often not recommended for a highly cohesive and low-coupling system
\cite{GlobalVariablesAreBad, GlobalVariablesAreEvil},
so dropping this part of the protection when it cannot be solved is generally not a big deal. 

\subsection{Extend the binary rewriting tool into distributed system}
Our work will explore two areas related to distributed systems, how to support legacy distributed
programs and enable tools to run on distributed systems.

\textbf{Support legacy distributed programs}.
These distributed legacy programs tend to have more complex features
than to ordinary programs.
For example, OpenMP \cite{Dagum1998OpenMPAI} and MPI+OpenMP \cite{Klinkenberg2020CHAMELEONRL},
will use mechanisms such as semaphores, message communication, etc., which cause problems
for both the identification and transformation of our tools.
We will explore and tackle these challenges during our research process.

\textbf{Tools run on distributed systems}.
How to run our tools on a distributed platform is a complex work.
DQEMU \cite{Zhao2020DQEMUAS} achieves a distributed dynamic binary translation system.
It discusses the implementation issues and performance optimization, including
data coherence protocol, locking mechanism, system calls, and remote thread migration.
We can take the idea of DQEMU and modify our tools to run on a distributed system.

\subsection{Optimize the above system by software-hardware co-optimization}
Whether running secret code in enclaves or using rewriting tools for transformation,
they both introduce a significant performance overhead.
Many studies investigated the overhead of trusted computing and 
the corresponding optimization method, including avoiding enclave switches
\cite{Tian2018SwitchlessCM} and reduceing page swaps \cite{Orenbach2017EleosEO, Taassori2018VAULTRP}.
But these optimizations may be challenging to implement in our tools because it is hard
to change the program's original behavior. 
In addition, binary rewriting also faces significant performance overhead, and existing
software optimizations are limited in dealing with these issues \cite{Kim2003HardwareSF}.

Some studies have proposed several ideas for hardware-assisted acceleration, such as
shared memory \cite{Jiang2022CRONUSFS}.
So our future work will summarize the existing optimization and search the performance
bottlenecks through performance analysis tools, such as Perf, VTune, etc.
After we obtain the bottlenecks of the performance, we can summarize the common characteristics
and design/modify some hardware modules to speed up our tools and programs similar to
the PIE \cite{Schneider2021PIEAP}.